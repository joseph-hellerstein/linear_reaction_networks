%%
%% Copyright 2022 OXFORD UNIVERSITY PRESS
%%
%% This file is part of the 'oup-authoring-template Bundle'.
%% ---------------------------------------------
%%
%% It may be distributed under the conditions of the LaTeX Project Public
%% License, either version 1.2 of this license or (at your option) any
%% later version.  The latest version of this license is in
%%    http://www.latex-project.org/lppl.txt
%% and version 1.2 or later is part of all distributions of LaTeX
%% version 1999/12/01 or later.
%%
%% The list of all files belonging to the 'oup-authoring-template Bundle' is
%% given in the file `manifest.txt'.
%%
%% Template article for OXFORD UNIVERSITY PRESS's document class `oup-authoring-template'
%% with bibliographic references
%%

%%%CONTEMPORARY%%%
\documentclass[unnumsec,webpdf,contemporary,large]{oup-authoring-template}%
%\documentclass[unnumsec,webpdf,contemporary,large,namedate]{oup-authoring-template}% uncomment this line for author year citations and comment the above
%\documentclass[unnumsec,webpdf,contemporary,medium]{oup-authoring-template}
%\documentclass[unnumsec,webpdf,contemporary,small]{oup-authoring-template}

%%%MODERN%%%
%\documentclass[unnumsec,webpdf,modern,large]{oup-authoring-template}
%\documentclass[unnumsec,webpdf,modern,large,namedate]{oup-authoring-template}% uncomment this line for author year citations and comment the above
%\documentclass[unnumsec,webpdf,modern,medium]{oup-authoring-template}
%\documentclass[unnumsec,webpdf,modern,small]{oup-authoring-template}

%%%TRADITIONAL%%%
%\documentclass[unnumsec,webpdf,traditional,large]{oup-authoring-template}
%\documentclass[unnumsec,webpdf,traditional,large,namedate]{oup-authoring-template}% uncomment this line for author year citations and comment the above
%\documentclass[unnumsec,namedate,webpdf,traditional,medium]{oup-authoring-template}
%\documentclass[namedate,webpdf,traditional,small]{oup-authoring-template}

%\onecolumn % for one column layouts

%\usepackage{showframe}

\graphicspath{{Fig/}}

% line numbers
%\usepackage[mathlines, switch]{lineno}
%\usepackage[right]{lineno}

\theoremstyle{thmstyleone}%
\newtheorem{theorem}{Theorem}%  meant for continuous numbers
%%\newtheorem{theorem}{Theorem}[section]% meant for sectionwise numbers
%% optional argument [theorem] produces theorem numbering sequence instead of independent numbers for Proposition
\newtheorem{proposition}[theorem]{Proposition}%
%%\newtheorem{proposition}{Proposition}% to get separate numbers for theorem and proposition etc.
\theoremstyle{thmstyletwo}%
\newtheorem{example}{Example}%
\newtheorem{remark}{Remark}%
\theoremstyle{thmstylethree}%
\newtheorem{definition}{Definition}

\begin{document}

\journaltitle{Journal Title Here}
\DOI{DOI HERE}
\copyrightyear{2022}
\pubyear{2019}
\access{Advance Access Publication Date: Day Month Year}
\appnotes{Paper}

\firstpage{1}

%\subtitle{Subject Section}

\title[Notes on Network Approximations]{Approximations}

\author[1,$\ast$]{First Author}
\author[2]{Second Author}
\author[3]{Third Author}
\author[3]{Fourth Author}
\author[4]{Fifth Author\ORCID{0000-0000-0000-0000}}

\authormark{Author Name et al.}

\address[1]{\orgdiv{Department}, \orgname{Organization}, \orgaddress{\street{Street}, \postcode{Postcode}, \state{State}, \country{Country}}}
\address[2]{\orgdiv{Department}, \orgname{Organization}, \orgaddress{\street{Street}, \postcode{Postcode}, \state{State}, \country{Country}}}
\address[3]{\orgdiv{Department}, \orgname{Organization}, \orgaddress{\street{Street}, \postcode{Postcode}, \state{State}, \country{Country}}}
\address[4]{\orgdiv{Department}, \orgname{Organization}, \orgaddress{\street{Street}, \postcode{Postcode}, \state{State}, \country{Country}}}

\corresp[$\ast$]{Corresponding author. \href{email:email-id.com}{email-id.com}}

\received{Date}{0}{Year}
\revised{Date}{0}{Year}
\accepted{Date}{0}{Year}

%\editor{Associate Editor: Name}

%\abstract{
%\textbf{Motivation:} .\\
%\textbf{Results:} .\\
%\textbf{Availability:} .\\
%\textbf{Contact:} \href{name@email.com}{name@email.com}\\
%\textbf{Supplementary information:} Supplementary data are available at \textit{Journal Name}
%online.}

\abstract{This is the abstract.
}
\keywords{keyword1, Keyword2, Keyword3, Keyword4}

% \boxedtext{
% \begin{itemize}
% \item Key boxed text here.
% \item Key boxed text here.
% \item Key boxed text here.
% \end{itemize}}

\maketitle


\section{Previous Theorectical Approach}
\subsection{Notation}
\begin{itemize}
\item $\alpha(s), \beta(s)$ are polynomials in $1/s$
\item $D(s)$ is the denominator of transfer function
\item $G(s)$ is the transfer function from $S_1$ to $S_M$
\item $j$ indexes reactions
\item $J_p = |p|$ is the number of reactions in the spath $p$
\item $\bar{J} = max_{p \in P} J_p$
\item $k_{m_1 m_2}$ is the kinetic constant for the reaction $S_{m_1} \rightarrow S_{m_2}$
\item $m$ indexes species
\item $M$ is the number of species
\item $N(s)$ is the numerator of transfer function
\item $p$ indexes spaths
\item $P$ is the set of sequential paths from $S_1$ to $S_M$
\item $s$ is the Laplace variable
\item $S$ are chemical species
\end{itemize}

\subsection{Details}
Consider a uni-uni, mass action reaction network. That is, all reactions have a single reactant and a single product.

The species are $S_m (s)$, $1 \leq m \leq M$. Reaction $S_i \rightarrow S_j$ has kinetics $k_{ij} S_i$. We assume that there is at most one reaction from $S_i$ to $S_j$, which is not a limitation since if there are multiple, their kinetic constants can be added to produce a single reaction. We want to construct $G(s)$, the transfer function with $S_1$ as input and $S_M$ as output.

A \textbf{forward sequential path (fpath)} $p$ from $S_1$ to $S_M$ is a sequence of $J$ reactions such that:
  1. There is a single reaction $S_1 \rightarrow S_{m_1}$.
  1. There is a single reaction $S_{m_{J}} \rightarrow S_M$.
  1. For every product $S_{m_j}$, except for $S_M$, there is a single reaction $S_{m_{j-1}} \rightarrow S_{m_j}$.
Note that if there are loops a path may be of infinite length.

A \textbf{backward sequential path (bpath)} $p$ from $S_M$ to $S_m$ is a sequence of $J$ reactions such that:
  1. There is a single reaction $S_M \rightarrow S_{m_1}$.
  1. For every product $S_{m_j}$, there is a single reaction $S_{m_{j-1}} \rightarrow S_{m_j}$, except for $S_{J_p}$ if $J_p$ is finite.

There are two types of bpath. In a \textbf{bpath I}, $J_p < \infty$ and there is a single reaction $S_{m_j} \rightarrow S_M$. In a **bpath II**, $S_M$ never appears as a product. Note that because of loops, $|P_I|$ may be infinite.

A \textbf{sequencial path (spath)} is either a fpath or a bpath.

We want to find an expression for $\dot{S}_M$ in terms of $S_1$.
Note that
\begin{eqnarray}
\dot{S}_M
& \equiv & s S(s)_M \\
& = & \sum_m^M (k_{mM} - k_{Mm}) S_{m}(s) \\
& = & \alpha(s) S_1 (s) - \beta(s) S_M(s)
\end{eqnarray}
where $\alpha(s), \beta(s)$ are polynomials in $\frac{1}{s}$. Note that the transfer function from $S_1$ to $S_M$ is
\begin{eqnarray}
G(s) &=& \frac{N(s)}{D(s)} \\
& = & \frac{s^{\bar{J}}\alpha(s)}{s^{\bar{J}} (s + \beta(s))}
\end{eqnarray}
where $\bar{J} = max_{p \in P} J_p$ for $P$ the set of spaths starting at $S_1$ or $S_M$ and the length of spath $p$ is $J_p$. Note that the order of the numerator is always smaller than the denominator and so we will obtain a proper (feasible) transfer function.

If there are loops in the reaction network, then $\bar{J}$ is infinite. This has two implications. First, there are constraints on the kinetic constants to ensure stability. Second, we may need to choose a maximum value for $\bar{J}$ to calculate it in practice. We return to this later.

For now, we assume that $\bar{J}$ is finite, and focus on $\alpha(s)$ and $\beta(s)$. $\alpha(s)$ relates to the synthesis of $S_M$ from $S_1$. Note that some synthesis may be the result of spaths.

First consider the reaction $S_1 \rightarrow S_M$ with kinetic constant $k_{1M}$. Thus, $\alpha(s)$ has a term $k_{1M} s^0 = k_{1M}$.
Now consider the spath of length 2.
\begin{eqnarray}
S_1 & \rightarrow & S_m \\
S_m & \rightarrow & S_M . \\
\end{eqnarray}
This spath implies that $\frac{d S_M}{d t}$ increases with $k_{mM} S_{mM}$ and $\frac{d S_m}{d t}$ increases with $k_{1m} S_1$. So, $\frac{k_{mM}k_{1m}}{s}$ is a term in $\alpha(s)$. Generalizing, for an spath $p$ with $|p|$ reactions from $S_1$ to $S_M$, there is a term $\frac{\prod_j k_{j_l j_r}}{s^J}$, where $j_l$ indexex the reactant species and $j_r$ indexes the product species. Summing across each spath $p$ in $P$, the spaths from $S_1$ to $S_M$
\begin{eqnarray}
\alpha(s) = \sum_{p \in P} \frac{\prod_{j \in p} k_{j_l j_r}}{s^{J_p}}
\end{eqnarray}
where $J_p = |P|$. Note that this sum can be infinite.

Now consider $\beta(s)$. In all cases we remove $S_M$ at a rate specified by the first reaction in the bpath. However, we may additionally add $S_M$ for a bpath type I.
\begin{eqnarray}
\beta(s)  =
&
\sum_{p \in P_I }
 \frac{\prod_{j \in p} k_{j_l j_r}}{s^{J_p}}
- \sum_{m} \frac{k_{M, m}}{s} \\
 \\
\end{eqnarray}
Note that $s^{\bar{J}} (s - \beta(s))$ is the characteristic equation of this system. So if the first summation of $\beta(s)$ is non-zero, the system is unstable by Routh-Hurwitz since there will be coefficients of $s^n$ with both positive and negative signs.

Notes
1. The characteristic equation has one nonzero pole and $\bar{J} -1$ poles at zero.
1. The solution has the form $\sum_{n=0}^{\bar{J}} c_n t^n e^{-pt}$, where $p$ is the pole. How well can this approximate an arbitrary function?

\end{document}
