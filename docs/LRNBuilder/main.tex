%%
%% Copyright 2022 OXFORD UNIVERSITY PRESS
%%
%% This file is part of the 'oup-authoring-template Bundle'.
%% ---------------------------------------------
%%
%% It may be distributed under the conditions of the LaTeX Project Public
%% License, either version 1.2 of this license or (at your option) any
%% later version.  The latest version of this license is in
%%    http://www.latex-project.org/lppl.txt
%% and version 1.2 or later is part of all distributions of LaTeX
%% version 1999/12/01 or later.
%%
%% The list of all files belonging to the 'oup-authoring-template Bundle' is
%% given in the file `manifest.txt'.
%%
%% Template article for OXFORD UNIVERSITY PRESS's document class `oup-authoring-template'
%% with bibliographic references
%%

%%%CONTEMPORARY%%%
\documentclass[unnumsec,webpdf,contemporary,large]{oup-authoring-template}%
%\documentclass[unnumsec,webpdf,contemporary,large,namedate]{oup-authoring-template}% uncomment this line for author year citations and comment the above
%\documentclass[unnumsec,webpdf,contemporary,medium]{oup-authoring-template}
%\documentclass[unnumsec,webpdf,contemporary,small]{oup-authoring-template}

%%%MODERN%%%
%\documentclass[unnumsec,webpdf,modern,large]{oup-authoring-template}
%\documentclass[unnumsec,webpdf,modern,large,namedate]{oup-authoring-template}% uncomment this line for author year citations and comment the above
%\documentclass[unnumsec,webpdf,modern,medium]{oup-authoring-template}
%\documentclass[unnumsec,webpdf,modern,small]{oup-authoring-template}

%%%TRADITIONAL%%%
%\documentclass[unnumsec,webpdf,traditional,large]{oup-authoring-template}
%\documentclass[unnumsec,webpdf,traditional,large,namedate]{oup-authoring-template}% uncomment this line for author year citations and comment the above
%\documentclass[unnumsec,namedate,webpdf,traditional,medium]{oup-authoring-template}
%\documentclass[namedate,webpdf,traditional,small]{oup-authoring-template}

%\onecolumn % for one column layouts

%\usepackage{showframe}

\graphicspath{{Fig/}}

% line numbers
%\usepackage[mathlines, switch]{lineno}
%\usepackage[right]{lineno}
% Package used  
\usepackage{graphicx}
\usepackage{hyperref}
\usepackage{subcaption}
\usepackage{fancyvrb}
\usepackage{xcolor}
\hypersetup{
    colorlinks,
    linkcolor={red!50!black},
    citecolor={blue!50!black},
    urlcolor={blue!80!black}
}

%\usepackage[style=authoryear, maxbibnames=2]{biblatex}
%\addbibresource{reference.bib}
%\AtEveryBibitem{\clearfield{month}}
%\AtEveryBibitem{\clearfield{day}}
%\AtEveryBibitem{\clearfield{doi}}
%\AtEveryBibitem{\clearfield{issn}}
%\AtEveryBibitem{\clearfield{pages}}

\theoremstyle{thmstyleone}%
\newtheorem{theorem}{Theorem}%  meant for continuous numbers
%%\newtheorem{theorem}{Theorem}[section]% meant for sectionwise numbers
%% optional argument [theorem] produces theorem numbering sequence instead of independent numbers for Proposition
\newtheorem{proposition}[theorem]{Proposition}%
%%\newtheorem{proposition}{Proposition}% to get separate numbers for theorem and proposition etc.
\theoremstyle{thmstyletwo}%
\newtheorem{example}{Example}%
\newtheorem{remark}{Remark}%
\theoremstyle{thmstylethree}%
\newtheorem{definition}{Definition}

\begin{document}

\journaltitle{Journal Title Here}
\DOI{DOI HERE}
\copyrightyear{2022}
\pubyear{2019}
\access{Advance Access Publication Date: Day Month Year}
\appnotes{Paper}

\firstpage{1}

%\subtitle{Subject Section}

\title[Modular Reaction Networks]{A Mathematical Framework for Building Modular Reaction Networks}

\author[1,$\ast$]{First Author}


\authormark{Author Name et al.}

\address[1]{\orgdiv{Department}, \orgname{Organization}, \orgaddress{\street{Street}, \postcode{Postcode}, \state{State}, \country{Country}}}


\corresp[$\ast$]{Corresponding author. \href{email:email-id.com}{email-id.com}}

\received{Date}{0}{Year}
\revised{Date}{0}{Year}
\accepted{Date}{0}{Year}

%\editor{Associate Editor: Name}

%\abstract{
%\textbf{Motivation:} .\\
%\textbf{Results:} .\\
%\textbf{Availability:} .\\
%\textbf{Contact:} \href{name@email.com}{name@email.com}\\
%\textbf{Supplementary information:} Supplementary data are available at \textit{Journal Name}
%online.}

\abstract{Most reaction networks are nonlinear in the sense that their dynamics are described by a system of nonlinear differential equations. In contrast, linear reaction networks (LRN) have dynamics that are described by a system of linear of differential equations. LRNs have appeal because they have an exact solution in the time domain, and so are amenable to use in systems for designing and/or modification to achieve desired goals related to stability, oscillations, settling times, and regulation. Herein, we on single-input, single-output (SISO) linear reaction networks (SLRN).
Key results are:}
\keywords{keyword1, Keyword2, Keyword3, Keyword4}

% \boxedtext{
% \begin{itemize}
% \item Key boxed text here.
% \item Key boxed text here.
% \item Key boxed text here.
% \end{itemize}}

\maketitle


\section{Introduction}
Here is a citation (\cite{al2002handgrip}).
\begin{enumerate}

%%%%%%%%%%%
\item
Motivation
\begin{enumerate}
\item
Modules are a fundamental concept in electrical, mechanical and software engineering.
\item
Define characteristics of modules (Gennari): inputs, outputs, behaviors.
By behaviors, we mean the relationship between the time course
of inputs (e.g., glucose concentration) and the
time course of outputs (e.g., the
concentration of ATP).
\item
Example of predicting time course behaviors.
\item
A module has well defined inputs and outputs, and there is a clear specification as to how inputs are transformed into outputs. A system is assembled from modules that only interact through their inputs and outputs. Further, there are well defined ways to combine modules that result in new modules with predictable behaviors.
Modularity facilitates analysis by
attributing system behaviors to the behaviors of modules
and the interactions between modules.
Modularity facilites design by providing a ``building block" approach
to constructing complex systems with desired behaviors.
\item
For example, ...
\end{enumerate}
\item
There have been many efforts to define modules for engineering biological systems.
\begin{enumerate}
\item BioBricks (well defined parts, but not modules)
\item Gennarri (benefits of violating modularity in models)
\end{enumerate}
%%%%%%%%%%%
\item Scope and Contributions.
Although there has been considerable progress in the development of standard
biological ``parts", there has been considerably less progress in the
development of biological modules with predictable behaviors.
In particular, we know of no work that defines biological modules
that can be composed in a way that results in predictable time course behaviors.

Our contribution is theoretical in that we do not build biological components.
Rather, we use the abstraction of a reaction network to
describe biological modules and systems.
By a reaction, we mean the conversion of one or more 
biological entities (e.g., chemical species, cells)
into other biological entities with a flux specified by a rate law
that is expressed
as an algebraic function of the concentration of biological entities.
Our approach is limited in many ways since it
does not address the details of particular biochemical environments.

Herein, we take a first step towards predicting defining modules
and module compositions with predictable beahviors.
We limit ourselves in two ways.
First, we only consider biological systems whose kinetics can be described
using systems of linear differential equations.
Although restrictive in a mathematical sense, our results are
still of considerable interest since it is common to use
such linear approximations to approximate deterministic, nonlinear systems.
A second restriction is that we only consider single input, single output (SISO)
systems.
Extensions to multiple input, multiple output (MIMO) systems are bit more mathematical,
and are deferred to future work.
Even so, our results here can be applied to MIMO systems of modest size complexity
by constructing a MIMO system from
multiple SISO systems.


Our contributions are:
\begin{itemize}
\item
Definition of a {\bf modular reaction network (MRN)} provides for predicting
composition of MRNs.
\item
Specification of several techniques for composing MRNs and specification
of behavior.
\end{itemize}

\begin{enumerate}
\item
retroactivity
\end{enumerate}

%%%%%%%%%%%
\item Contributions
\end{enumerate}



\section{Methods}\label{methods}
\begin{enumerate}
\item
Motivation: 
Arbitrary real-valued functions can be approximated by a fourier transform,
and the fourier transform is a subset of the Laplace Transform (where the real part is 0).
\item
This analysis is limited to SISO and transfer functions that
are rational polynomials of $s$ with real coefficients.
\item
Evaluations: (a) can approximate complex simulations; (b) composed models have predictable transfer functions.
\item
Create a {\em sufficient} set of operations such that can
approximate complex simulations.
\end{enumerate}



\section{Results}\label{results}
\subsection{Modular Reaction Networks}

We use subscripted and superscripted $S$ to indicate chemical species. Where there is no confusion, we use the same symbol for its concentration as well. In contexts where there may be confusion between the name and its concentration, we explicitly indicate a concentration using square brackets. For example, $[S_0]$ is the concentration of species $S_0$.

A {\bf reaction} is a four tupleexpressed as $S_I \xrightarrow{k} m S_O$
indicates the transformation of mass from species $S_I$ to $m$ molecules of $S_O$
using mass action with the kinetic constant $k$. If $k < 0$,
then $S_O$ is transformed into $S_I$; that is, $S_O \xrightarrow{-k} S_I$.
If $S_I = \emptyset$ or $S_O = \emptyset$, then this a boundary reaction.
For an input boundary, $k_I$ is a fixed rate.
If $S_O = \emptyset$, then this is a degradation reaction.

Let $\mathcal{R}$ be a set of reactions, and let
$\mathcal{S}$ be the set of all reactants and products in $\mathcal{R}$.
$S_I$ is the input species for the network.
We require that $S_I$ never appear as a product in the reaction network.
$S_O$ is the output species.
We require that $S_O$ never appears as a reactant in the reaction network.
A {\bf SISO Linear Reaction Network (SLRN)} is
a 5-tuple
$<\mathcal{R}, S_I, S_O, k_I, k_O>$ where:
\begin{itemize}
    \item $\mathcal{R}$ is a set of reactions.
    \item 
    $S_I, S_O \in \mathcal{S}$ are input and output species.
    \item $k_I S_I$ is the rate at which the input is consumed
    by the SLRN.
    \item $k_O S_O$ is the rate at which the output is degraded by the SLRN.
\end{itemize}
Sometimes, we work with multiple SLRNs. Components of networks are indicated by a
capital letter superscript.
For example, for network $A$, we have the 5-tuple
$<\mathcal{R}^A, S^A_I, S^A_O, k^A_I, k^A_O>$.

\subsubsection{Sequential Reaction Network}

We start with a sequential network with $N$ stages.
There are $N$ stages with an input and output. The input is $S_0$ and the output is $S_N$. Formally, we have
\begin{eqnarray}
s S_n (s) & =&  k_{n-1} S_{n-1} (s) - (k_n + k^{\prime}) S_n (s) ,~ n>0
\end{eqnarray}

From this, we derive the transfer function for the network.
\begin{align}
S_n (s) & = & \frac{k_{n-1} S_{n-1}}{s + k_n + k^{\prime}_n} \nonumber \\
S_1 (s) & = & \frac{k_0 S_0}{s + k_1 + k^{\prime}_1} \nonumber \\
S_2 (s) & = & \frac{k_1 S_1}{s + k_2 + k^{\prime}_2} \nonumber \\
  & = & \frac{k_1 k_0 S_0}{(s + k_2 + k^{\prime}_2)(s + k_1 + k^{\prime}_1)}
\end{align}

\begin{eqnarray}
S_N (s) & = & \frac{S_0 (s) \prod_{n=0}^{N-1} k_n }{\prod_{n=1}^N(s + k_n + k^{\prime}_n)} \nonumber \\
    & = &  \frac{ k_0 }{s  +  k^{\prime}_N}  
    \frac{\prod_{n=1}^{N-1} k_n }{\prod_{n=1}^{N-1}(s + k_n + k^{\prime}_n)} S_0(s) \nonumber \\
G (s) & = &  \frac{ k_0 }{s  +  k^{\prime}_N}  
    \frac{\prod_{n=1}^{N-1} k_n }{\prod_{n=1}^{N-1}(s + k_n + k^{\prime}_n)}
\end{eqnarray}
where $k_N = 0= k^{\prime}_0$.

\begin{figure}
        \centering
         \includegraphics[scale=0.4]{figures/sequential_network.png}
          \caption[]{An $N$ stage sequential network.}
         \label{fig:sequential_network}
\end{figure}


%%%%%%%%%%%%%%%%%
\subsection{Operations on SISO Reaction Networks}
These are operations applied to SLRNs and produce a new SLRN.
In the sequel, $A, B$ are SLRNs.
%%%%%%%%%%%%%%%%%%%%%%%%%%%%%%%%

\subsubsection{concatenate}
\begin{figure}
         \centering
         \includegraphics[scale=0.3]{figures/concatenate.png}
          \caption[]{Concatenate operation on networks $A$ and $B$.}
         \label{fig:concatenate}
\end{figure}

{\bf Concatenate} is an asymmetric, binary operation.
Let $A, B$ be such that $\mathcal{S}^A \bigcap \mathcal{S}^B =
\{ S^A_O \} = \{ S^B_I \}$. 
Concatenate applied these SLRNs, denoted by $A \circ  B$,
and produces
the following SLRN:
\begin{itemize}
\item $S_I =S^A_I$
\item $S_O = S^B_O$
\item $\mathcal{R} = \mathcal{R}^A \bigcup \mathcal{R}^B$
\item $k_I = k^A_I$
\item $k_O = k^B_O$
\end{itemize}

Transfer function of the {\em concatenate} operation.
$$G(s) = G^A(s) G^B(s)\frac{s + k^A_O}{s+ k^B_I + k^A_O}$$.

%%%%%%%%%%%%%%%%%
\subsubsection{Branchjoin}
\begin{figure}
         \centering
         \includegraphics[scale=0.3]{figures/branchjoin.png}
          \caption[]{Branchjoin operation on networks $A$ and $B$.}
         \label{fig:branchjoin}
\end{figure}

{\bf Branchjoin} is a symmetric, binary operation on SLRNs $A, B$ for which
$S^A \bigcap S^B = \emptyset$.
$branchjoin(A, B; k_{1a}, k_{1b}, k_{2a}, k_{2b}, k_3)$
produces:
\begin{itemize}
\item $S_I$, a new species
\item $S_O$, a new species
\item $\mathcal{R} = \mathcal{R}^A \bigcup \mathcal{R}^B$ and the following reactions.
\begin{itemize}
\item
$S_I \xrightarrow{k_{1a}} S^A_I$
\item $S_I \xrightarrow{k_{1b}} S^B_I$
\item
$S^A_O \xrightarrow{k_{2a}} S_O$
\item
$S^B_O \xrightarrow{k_{2b}} S_O$
\item
$S_O \xrightarrow{k_3} \emptyset$
\end{itemize}
\item $k_I = k_{1a} + k_{1b}$
\item $k_O = k_3$
\end{itemize}

Transfer function of the {\em branchjoin} operation.
Let $N_1, N_2$ be reaction networks with transfer functions $G^A(s), G^B(s)$. Let $H(s)$ be the transfer function for $N^A + N^B$.

We begin by writing the state equations based on the flow diagram.
   \begin{eqnarray}
   s S^A_I (s) & = & k_{1a} S_I(s) - k^A_I S^A_I (s) \nonumber \\
   s S^B_I (s) & = & k_{1b} S_I(s) - k^B_I S^B_I (s) \nonumber \\
   S^A_O (s) & = &  S^A_I (s) G^A(s) \frac{s + k^A_O}{s + k^A_O + k_{2a}} \nonumber \\
   S^B_O (s) & = & S^B_I (s)  G^B(s) \frac{s + k^B_O}{s + k^B_O + k_{2b}} \nonumber \\
   sS_O (s) & = & k_{2a} S^A_O(s) + k_{2b} S^B_O(s) - k_3 S_O(s)
   \end{eqnarray}

The following are dervied from the state equations.
\begin{eqnarray}
S^A_I(s) & = & \frac{k_{1a} S_I (s)}{s + k^A_I} \nonumber \\
S^B_I(s) & = & \frac{k_{1b} S_I (s)}{s + k^B_I} \nonumber \\
S^A_O & = & \frac{k_{1a} S_I (s)}{s + k^A_I} G^A(s) \frac{s + k^A_O}{s + k^A_O + k_{2a}} \nonumber\\
   & = & \frac{k_{1a}}{s + k^A_I} \frac{s + k^A_O}{s + k^A_O + k_{2a}}  G^A(s) S_I (s) \nonumber \\
S^B_O
   & = & \frac{k_{1b}}{s + k^B_I} \frac{s + k^B_O}{s + k^B_O + k_{2b}}  G^B(s) S_I (s) \nonumber \\
S_O (s) & = & \frac{k_{2a} S^A_O(s) + k_{2b} S^B_O(s) }{s + k_3} \nonumber \\
G(s) & = & \frac{k_{1a} k_{2a} (s + k^A_O) \beta(s) G^A(s) }{\alpha(s) \beta(s) (s + k_3)} \nonumber \\
& & +  \frac{k_{1b}  k_{2b} (s + k^B_O)\alpha(s) G^B(s) }{\alpha(s) \beta(s) (s + k_3)} \\
\end{eqnarray}

$\alpha(s) = (s + k^A_I) (s + k^A_O + k_{2a})$ and
$\beta(s) = (s + k^B_I) (s + k^B_O + k_{2b})$.

%%%%%%%%%%%%%%%%%
\subsubsection{Scale}
\begin{figure}
         \centering
         \includegraphics[scale=0.4]{figures/scale.png}
          \caption[]{Scale operation on network $A$.}
         \label{fig:scale}
\end{figure}
{\bf Scale} is a unary operation.
$scale(A, k, m)$ produces 
\begin{itemize}
\item $S_I = S^A_O$
\item $S_O$, a new species
\item $R = \mathcal{R}^A \bigcup \{S_O \xrightarrow{k} \frac{m}{k} S_O\}$
\item $k_I = k$
\item $k_O = 0$
\end{itemize}

Transfer function of the {\em scale} operation.
Scale is a concatenation.
The transfer function for the scale reaction is
$H(s) = \frac{km/k } {s}$, and the concatenation with $A$
has the transfer function
$G(s) = G^A(s) \frac{k^A_O}{k^A_O} H(s)$ and so
$G(0) = m$.


%%%%%%%%%%%%%%%%%
\subsubsection{Loop}

\begin{figure}
         \centering
         \includegraphics[scale=0.4]{figures/loop.png}
          \caption[]{Loop operation on network $A$.}
         \label{fig:concatenate}
\end{figure}
{\bf Loop} is a unary operation.
$loop(A, S_I, S_O, k_1, k_2, k)$ is a unary operation that produces
the SLRN:
\begin{itemize}
\item $S_I$ is a new species.
\item $S_O$ is a new species.
\item $\mathcal{R} = \mathcal{R}^A \bigcup$ the following
\begin{itemize}
\item $S_I \xrightarrow{k_1} S^A_I$
\item $S^A_O \xrightarrow{k} S_I$
\item $S^A_O \xrightarrow{k_2} S_O$
\end{itemize}
\item $k_I = k^A_I$
\item $k_O = k^A_O + k$
\end{itemize}

The following are the state equations of this system.
\begin{eqnarray}
s S_O (s) & = & k_3 S^A_O (s) - (k_4 + k_5) S_O \nonumber \\
S_O & = & \frac{k_3 S^A_O(s)}{s + k_4 + k_5} \\
s S^A_I(s) & = & k_2 X_I(s) - k^A_I S^A_I(s) \nonumber \\
S^A_I(s) & = & \frac{k_2 X_I(s)}{s + k^A_I} \\
s X_I (s) & = & k_5 S_O(s) + k_1 S_I (s) - k_2 X_I (s) \nonumber \\
X_I(s) & = & \frac{k_5 S_O(s) + k_1 S_I (s)}{s + k_2} \\
S^A_O(s) & = & S^A_I(s) G^A(s) \frac{s + k^A_O}{s + k^A_O + k_3}
\end{eqnarray}
We manipulate the state equations to derive the transfer function of
the loop operation.
This analysis is contained in
Fig.~\ref{fig:loop-derivation}

\begin{figure*}
\begin{eqnarray}
s S_O (s) & = & k_3   G^A(s) \frac{s + k^A_O}{ s + k^A_O + k_3}  S^A_I(s) - (k_4  + k_5) S_O (s) \nonumber \\
& = &
  k_2 k_3  G(s) \frac{s + k^A_O}{(s + k^A_O + k_3)(s + k^A_I)} X_I(s)
   -(k_4 S + k_5 )S_O (s) \nonumber \\
& = &
k_2 k_3  G^{\prime}(s)X_I(s) - (k_4 + k_5) S_O(s)  \nonumber \\
S_O & = &  \frac{k_2 k_3  G^{\prime}(s) X_I(s)}{s + k_4 + k_5  }  \nonumber \\
    & = &  \frac{k_2 k_3  G^{\prime}(s)}{s + k_4 + k_5  } \frac{k_5 S_O(s) + k_1 S_I (s)}{s + k_2} \nonumber \\
(s + k_4 + k_5)(s + k_2)S_O & = &
   k_2 k_3  G^{\prime}(s) (k_5 S_O(s) + k_1 S_I(s))
    \nonumber \\
S_O & = &  \frac{k_1 k_2 k_3  G^{\prime}(s) S_I(s)}{(s + k_4 + k_5)(s + k_2) + k_2 k_3 k_5 G^{\prime}(s)} \nonumber \nonumber \\
H(s) & = & \frac{k_1 k_2 k_3  G^{\prime}(s)}{(s + k_4 + k_5)(s + k_2) + k_2 k_3 k_5 G^{\prime}(s)} \nonumber \\
    & = & \frac{k_1 k_2 k_3  (s + k^A_O) G^A(s)}{(s + k^A_O + k_3)(s + k^A_I)(s + k_4 + k_5)(s + k_2) + k_2 k_3 k_5 (s + k^A_O)G(s)}
\end{eqnarray}
\caption{Loop derivations}\label{fig:loop-derivation}
\end{figure*}

%%%%%%%%%%%%%%%%%%%%%%%%%%%%%
\section{Discussion}\label{discussion}
\begin{enumerate}

%%%%%%%%%%%%%
\item
Principles
\begin{enumerate}
\item 
Adding poles and zeros
\item
Modular design of network so can account for retroactivity
\item
Technique for adjusting transfer functions for retroactivity
\end{enumerate}

\end{enumerate}

\section{Conclusions}

\section{Issues}
\begin{enumerate}
    \item Should I consider MIMO components? This seems possible, but it means more complex state models?
    \begin{enumerate}
    \item There should be separate $branch$ and $join$ operations.
    \item Inputs and outputs should be indexed, $S_{I_i}$, $S_{O_j}$.
    \end{enumerate}
    \item First develop the SISO theory?
\end{enumerate}

\bibliographystyle{plainnat}
\bibliography{reference}
\end{document}
